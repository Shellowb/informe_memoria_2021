
\newglossaryentry{latex}
{
    name=latex,
    description={Is a mark up language specially suited for scientific documents}
}

\newacronym{cadcc}{CADCC}{Centro de Alumnos del Departamento de Ciencias de la Computación}

\newacronym{dcc}{DCC}{Departamento de Ciencias de la Computación, de la Universidad de Chile}

\newacronym{ps}{CC5402}{CC5402 Proyecto de software} 

\newacronym{e}{CC6908}{CC6908 Introducción al trabajo de título}

\newacronym{f1}{CC6909}{CC6909 Trabajo de Titulo}

\newacronym{f2}{CC6910}{CC6910 Trabajo de Memoria de Titulo}

\newacronym{S}{\textit{Alumno S}}{alumno auto suficiente}

\newacronym{R}{\textit{Alumno R}}{alumno regular}

\newacronym{P}{\textit{Alumno P}}{alumno práctico}

\newacronym{UCD}{UCD}{User Centered Desing}

\newacronym{f}{FG}{\textit{Focus Group}}

\newglossaryentry{i*}{
    name=i*,
    description={El lenguaje de modelamiento i*, fue introducido como un marco de referencia orientado a modelar actores y objetivos. Consiste en un lenguaje de modelamiento junto con una serie de tecnicas que permiten analizar estos modelos. \cite{Dalpiaz2016}}
    }
    
\newglossaryentry{effectiveGUI}{
    name=Efective GUI,
    description= {Cuando se habla de \textit{Efective GUI} o interfaz gráfica efectiva para el usuario, se habla de una interfaz que está diseñada para simular o reemplazar una interacción humana, se acuña principalmente en el área de IA relacionada con los chatbots}
    }

\newglossaryentry{Telegram}{
    name=Telegram,
    description= {Telegram es un software gratiuto, multiplataforma, basado en la nube, de mensajería instantanea. El servicio además provee video encriptado de principio a fin, VoIP, envío de archivos y varias otras funciones. Fue lanzado para iOS en Agosto 14 2013 y Android en Octubre 2013 }
}

\newglossaryentry{Celery}{
    name=Celery,
    description= {Telegram es un software gratiuto, multiplataforma, basado en la nube, de mensajería instantanea. El servicio además provee video encriptado de principio a fin, VoIP, envío de archivos y varias otras funciones. Fue lanzado para iOS en Agosto 14 2013 y Android en Octubre 2013 }
}

\newglossaryentry{Django}{
    name=Django,
    description= {Telegram es un software gratiuto, multiplataforma, basado en la nube, de mensajería instantanea. El servicio además provee video encriptado de principio a fin, VoIP, envío de archivos y varias otras funciones. Fue lanzado para iOS en Agosto 14 2013 y Android en Octubre 2013 }
}

\newglossaryentry{ORM}{
    name=ORM,
    description= {Telegram es un software gratiuto, multiplataforma, basado en la nube, de mensajería instantanea. El servicio además provee video encriptado de principio a fin, VoIP, envío de archivos y varias otras funciones. Fue lanzado para iOS en Agosto 14 2013 y Android en Octubre 2013 }
}

\newglossaryentry{ODM}{
    name=ODM,
    description= {Telegram es un software gratiuto, multiplataforma, basado en la nube, de mensajería instantanea. El servicio además provee video encriptado de principio a fin, VoIP, envío de archivos y varias otras funciones. Fue lanzado para iOS en Agosto 14 2013 y Android en Octubre 2013 }
}

\newglossaryentry{REST}{
    name=REST,
    description={Representational State Transfer, architectural style for distributed hypermedia systems}
}
