%% Acronyms
\newacronym{cadcc}{CADCC}{Centro de Alumnos del Departamento de Ciencias de la Computación}

\newacronym{dcc}{DCC}{Departamento de Ciencias de la Computación, de la Universidad de Chile}

\newacronym{ps}{CC5402}{CC5402 Proyecto de software} 

\newacronym{e}{CC6908}{CC6908 Introducción al trabajo de título}

\newacronym{f1}{CC6909}{CC6909 Trabajo de Título}

\newacronym{f2}{CC6910}{CC6910 Trabajo de Memoria de Título}

\newacronym{S}{\textit{Alumno S}}{alumno autosuficiente}

\newacronym{R}{\textit{Alumno R}}{alumno regular}

\newacronym{P}{\textit{Alumno P}}{alumno práctico}

\newacronym{UCD}{UCD}{\textit{User Centered Design}}

\newacronym{f}{FG}{\textit{Focus Group}}

\newacronym{uml}{UML}{\textit{Unified Modeling Language}}


%% Glossaries
\newglossaryentry{i*}{
    name=i*,
    description={El lenguaje de modelamiento i*, fue introducido como un marco de referencia orientado a modelar actores y objetivos. Consiste en un lenguaje de modelamiento junto con una serie de tecnicas que permiten analizar estos modelos \cite{Dalpiaz2016}.}
    }
    
\newglossaryentry{effectiveGUI}{
    name=Efective GUI,
    description= {Cuando se habla de \textit{Efective GUI} o interfaz gráfica efectiva para el usuario, se habla de una interfaz que está diseñada para simular o reemplazar una interacción humana, se acuña principalmente en el área de IA relacionada con los chatbots.}
    }

\newglossaryentry{Telegram}{
    name=Telegram,
    description= {Telegram es un software gratiuto, multiplataforma, basado en la nube, de mensajería instantanea. El servicio además provee video encriptado de principio a fin, VoIP, envío de archivos y varias otras funciones. Fue lanzado para iOS en Agosto 14 2013 y Android en Octubre 2013.}
}

\newglossaryentry{Redis}{
    name=Redis,
    description= {Redis es un almacén de estructura de datos en memoria de código abierto (con licencia BSD) que se utiliza como base de datos, caché, intermediario de mensajes y motor de transmisión. Redis proporciona estructuras de datos como cadenas, hashes, listas, conjuntos, conjuntos ordenados con consultas de rango, mapas de bits, hiperloglogs, índices geoespaciales y flujos. Redis tiene replicación integrada, secuencias de comandos Lua, desalojo de LRU, transacciones y diferentes niveles de persistencia en el disco, y proporciona alta disponibilidad a través de Redis Sentinel y partición automática con Redis Cluster. }
}

\newglossaryentry{Celery}{
    name=Celery,
    description= {Las colas de tareas se utilizan como un mecanismo para distribuir el trabajo entre subprocesos o máquinas.
    La entrada de una cola de tareas es una unidad de trabajo denominada tarea. Los procesos de trabajo dedicados monitorean constantemente las colas de tareas para realizar nuevos trabajos.
    Celery se comunica a través de mensajes, generalmente utilizando un intermediario para mediar entre clientes y trabajadores. Para iniciar una tarea, el cliente agrega un mensaje a la cola, el intermediario luego entrega ese mensaje a un trabajador.
    Un sistema Celery puede constar de varios trabajadores y agentes, lo que da paso a una alta disponibilidad y escalamiento horizontal.}
}

\newglossaryentry{Django}{
    name=Django,
    description= {Django es un \textit{framework} web basado en Python de alto nivel que fomenta un desarrollo rápido y un diseño limpio y pragmático. Creado por desarrolladores experimentados, se ocupa de gran parte de las molestias del desarrollo web, por lo que puede concentrarse en escribir su aplicación sin necesidad de reinventar la rueda. Es gratis y de código abierto.}
}

\newglossaryentry{ORM}{
    name=ORM,
    description= {Telegram es un software gratiuto, multiplataforma, basado en la nube, de mensajería instantanea. El servicio además provee video encriptado de principio a fin, VoIP, envío de archivos y varias otras funciones. Fue lanzado para iOS en Agosto 14 2013 y Android en Octubre 2013.}
}

\newglossaryentry{ODM}{
    name=ODM,
    description= {Telegram es un software gratiuto, multiplataforma, basado en la nube, de mensajería instantanea. El servicio además provee video encriptado de principio a fin, VoIP, envío de archivos y varias otras funciones. Fue lanzado para iOS en Agosto 14 2013 y Android en Octubre 2013 }
}

\newglossaryentry{REST}{
    name=REST,
    description={Representational State Transfer, architectural style for distributed hypermedia systems}
}

\newglossaryentry{Mongodb}{
    name=Mongodb,
    description={MongoDB es un programa de base de datos orientado a documentos multiplataforma con código fuente disponible. Clasificado como un programa de base de datos NoSQL, MongoDB utiliza documentos similares a JSON con esquemas opcionales. MongoDB es desarrollado por MongoDB Inc. y tiene licencia bajo la Licencia pública del lado del servidor.}
}

\newglossaryentry{React}{
    name=React,
    description={React es una biblioteca de JavaScript front-end gratuita y de código abierto para crear interfaces de usuario basadas en componentes de interfaz de usuario. Lo mantiene Meta y una comunidad de desarrolladores individuales y empresas.}
}

\newglossaryentry{websocket}{
    name=websocket,
    description={WebSocket es un protocolo de comunicaciones informáticas que proporciona canales de comunicación full-duplex a través de una única conexión TCP. El protocolo WebSocket fue estandarizado por el IETF como RFC 6455 en 2011. La especificación actual se conoce como HTML Living Standard.}
}

\newglossaryentry{Djongo}{
    name=Djongo,
    description={Djongo proporciona un enfoque unificado para la interfaz de bases de datos. Es una extensión del framework Django ORM tradicional. Mapea objetos de python a documentos MongoDB, una técnica conocida popularmente como Mapeo de documentos a objetos u ODM.}
}

\newglossaryentry{Flower}{
    name=Flower,
    description={Flower is a web based tool for monitoring and administrating Celery clusters.}
}

