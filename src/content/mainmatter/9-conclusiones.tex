\chapter{Conclusiones}
\section{Discusión Final}
    \par Con esta memoria se busca seguir mejorando el sistema de mesa de ayuda y, de este modo, mejorar la experiencia de los alumnos durante los procesos académicos, particularmente el proceso de titulación. El objetivo general era \guillemotleft extender el sistema actual, agregando funcionalidades que permitan crear un sistema extensible, personalizado y confiable. El cual será actualizado principalmente
    por el \acrshort{cadcc}, con miras de integrar otros actores posteriormente, favoreciendo la continuidad de este servicio\guillemotright. Para esto se dividió el objetivo principal en 6 objetivos específicos:

    \par En concordancia con el objetivo \href{sec:obj-e}{1}, se analizó la solución existente y se determinaron que las modificaciones necesarias eran: Cambio de procesamiento condicional de la \textit{request}, la separación en capas de procesamiento de mensajes, así cómo la estandarización y modularización de las respuestas del sistema a través de la creación del \textit{Parser}, el \textit{Handler} y los objetos asociados a las acciones.
    \par Se reestructuró el código existente, lo que permitirá la adición de nuevas funcionalidades en el futuro, así cómo extender y mejorar las ya presentes.

    \par Se rediseñaron los modelos actuales, así mismo como las funcionalidades, para que el sistema fuera capaz de recibir las mejoras presentadas en concordancia con el objetivo \href{sec:obj-e}{2}.

    \par  Cumpliendo el objetivo \href{sec:obj-e}{4}, se desarrolló un modelo de suscripción personalizada: El que permite para cada proceso agregar opciones de subscripción a las que el usuario puede suscribirse, así mismo se creó un modelo de notificaciones, y data mínima del alumno para habilitar o deshabilitar estas funcionalidades. Se creó la opción de generar recordatorios a partir de la suscripción a las etapas de una cierta instancia de proceso académico
    \par A partir del diseño, la modificación e implementación
    de nuevas funcionalidades, y resolviendo los problemas de compatibilidad de las librerías, se asegura la integridad de los modelos de datos, del código y se permite la extensibilidad a través de la modularidad lógica, cumpliendo el objetivo \href{sec:obj-e}{4}
    \par Además se incluyeron explícitamente en el diseño las preferencias de los alumnos, y buscando funcionalidades relevantes para ellos, a través del diseño de los diagramas \gls{i*} y su posterior conversión a                                                                                                                                                                                                                                                                                                                                                                                                                                                                                                                                                                                                                                                                                                                                                                                                                                                                                                      , y funcionalidades del sistema. Por esta misma razón se creó un sistema dual que permite tanto la interacción a través de botones cómo de comandos.

    \par Sin embargo, los problemas de compatibilidad, sumado a lo extenso de las reestructuraciones, no permitieron agregar más funcionalidades de suscripción para los otros contenidos valorados por los alumnos y tampoco añadir componentes de confiabilidad en la información. Aunque se creó la segmentación suficiente como para que sea sencillo añadir estas funciones en el futuro.

    \par Se añadieron además al proyecto, documentación necesaria para levantar el proyecto, tener nociones clave de los usuarios y sus preferencias, establecer un marco conceptual que permita darle una dirección estratégica al proyecto en el futuro. Permitiendo la mantención, la mejora y el desarrollo continuo por alumnos del dcc.

    \par En resumen se extendió el sistema actual agregando los módulos de \textit{parser}, \textit{handler}, y \textit{actions}, se modificaron y crearon modelos de datos para proveer funcionalidades de suscripción, se corrigieron errores de compatibilidad en la base de datos, se agregó un módulo para el procesamiento de tareas asíncronas. Se añadieron opciones de configuración, además de recordatorios programables. Además de documentación y nociones clave en desarrollo y mantenibilidad del proyecto, permitiendo crear un sistema, extensible, personalizado y confiable.

    \par A partir del trabajo se lograron varios aprendizajes. Primero, extender el trabajo de otra persona, si este esta mal documentado es inviable el trabajo práctico dónde los tiempos son acotados y se esperan resultados en plazos más o menos fijos. Por eso se hacen necesarias buenas prácticas de programación para extender los sistemas existentes, sobre todo cuando las personas que van a ir rotando constantemente.

    \par Las soluciones que dependen de librerías o diseños de terceros son muy vulnerables a quedar deprecadas en el corto o mediano plazo. Sobre todo si son tecnologías de uso extensivo. Esto hace que los encargados de un proyecto necesariamente deban conocer las dependencias del sistema y su rol en el mismo, y mantenerse al tanto de los cambios que estas implican. Por otro lado, mantenerse aislado de las actualizaciones que las librerías externas puedan tener, hace que el código se degrade aceleradamente, porque se va quedando sin recursos para interactuar con nuevas tecnologías. Esto implica que una empresa se puede quedar atrás muy rápido si no hace un adecuado balance entre mantenerse al día con las tecnologías usadas y reestructurar el código para ir asegurando su integridad cada cierto tiempo.
    \par La verdad es que un balance complejo, que la muchas de las instituciones no sabe llevar, porque si algo funciona, no se cambia no se toca, el problema es que en el mundo del software, algo rara vez permanece estático por mucho tiempo, a menos que haya sido abandonado. Son las constantes y crecientes mejoras las que van haciendo a los sistemas más robustos. En ese sentido la conclusión más importante que se obtiene de este trabajo, en el área de la mantención de software, es que se necesita adquirir un balance entre las herramientas que se usan y la mantención que estas requieren, y muchas veces este balance depende de un equipo tan calificado como en el área de innovación. Esta es una de las razones por las que muchas empresas prefieren externalizar todo aquello que no es de forma directa parte de su negocio, porque se vuelve muy difícil y costoso mantener las soluciones existentes.

    \par Se hace muy útil, conocer herramientas diversas a la hora de desarrollar una solución, porque esto puede reducir los tiempos de desarrollo considerablemente. Aunque siempre que se incluya una solución externa, es muy necesario evaluar el impacto de aprendizaje, mantención y capacidad de desarrollo que serán necesarias para mantener la solución vigente.

    \par Otra de las conclusiones que se saca del trabajo realizado, es que el software basado en usuarios, en el que se recogen las preferencias de cada usuario individual es complejo. Está complejidad se puede apreciar en que: Requieren una gran cantidad de diseño. Dependen de una gran cantidad de datos. Realizar una tarea de la forma en que cada usuario espera es muy costoso en tiempo para la mayoría de las empresas pequeñas. Sin embargo, también se da cuenta, que si se hace un análisis lo suficiente claro respecto a los objetivos y valores del usuario es posible obtener buenos resultados de diseño y acortar la implementación, lo que puede ser una alternativa para soluciones de menor escala.

    \par Finalmente, para que un trabajo tenga continuidad es necesario que el proyecto de software desarrollado tenga su propia lógica, que pueda ser continuada por personas que no necesariamente están en contacto con los creadores. Así mismo, se ve la importancia del trabajo en equipo para desarrollar soluciones complejas y completas, no solo para realizarlas más rápido, sino porque el resultado es más duradero en el tiempo. Es por eso, que a juicio del alumno, el mundo opensource es todavía mayoría en lo que se refiere a software, porque para que las soluciones permanezcan y evolucionen en el tiempo deben evolucionar las comunidades junto con ellas.

\section{Pasos a seguir}
    \par Queda pendiente una validación con los usuarios de las nuevas funcionalidades. Aunque se añaden conceptos clave para traer al diseño del sistema sus preferencias y objetivos. Así como un marco de decisión para la evaluación de nuevas funcionalidades del sistema.
    \par A partir de la memoria anterior, aún queda pendiente la puesta en marcha del sistema. Y su expansión a otros actores del departamento. Este punto es clave, y se espera que a través de los acuerdos llegados con el \acrshort{cadcc}, se puedan pasar a producción este año.
    \par El \textit{refactoring} del bot y su separación en capas de procesamiento a través del \textit{parser}, permite añadir nuevas formas de comunicación y procesamiento de información como el procesamiento en lenguaje natural propuesto en el trabajo anterior \cite{ARANCIBIA2021}
    \par la adición del módulo de tareas permite la adición de otros modelos de aprendizaje automático no supervisado. Uno de los propuestos en esta memoria es el desarrollar un sistema de notificaciones que pueda tomar inputs del usuario, inputs sociales anónimos y cambios en la información para determinar el mejor momento para interactuar con el alumno. Y se pueda adecuar al flujo de trabajo individual de cada usuario.
    \par También tanto la interfaz de \gls{Telegram} como la plataforma web, abren la posibilidad de investigaciones sobre de \textit{UX} y {UI}.
    \par por otro lado, la puesta en marcha del sistema es el paso clave, y se sugiere que cualquier nuevo proyecto, tenga como objetivo inicial asegurar la puesta en marcha del sistema. 
    \par Así mismo se recomienda extender la documentación y crear un equipo de soporte para la plataforma, que permita su funcionamiento independiente de los trabajos de título que puedan mejorarlo. Actualmente el \acrshort{cadcc} es un aliado clave en este proceso, pero se recomienda que más actores pudieran verse involucrados en la mejora del sistema.
