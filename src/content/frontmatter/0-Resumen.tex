\begin{resumen}
   \par El \acrfull{dcc}, ha tenido un aumento sustancial en la cantidad de alumnos que ingresan a la carrera. Esto implica que se generen una serie de desafíos en torno a los procesos académicos. Estos desafíos generan un aumento en la carga de los funcionarios y docentes, a su vez que hacen que no se pueda responder con la velocidad necesaria que requieren los cambios. Esto aumenta la incertidumbre por parte de los alumnos, sobre todo en la forma de abordar los diferentes desafíos de los procesos académicos. Este proyecto busca continuar la extensión de un sistema de mesa de ayuda, pensado para modernizar el trato efectivo entre alumnos, profesores y funcionarios. Mejorando su infraestructura y añadiendo funcionalidades clave para la continuidad y mejora del proyecto.

   \par Este proyecto se realizó en 4 fases. En primer lugar, se efectuó un nuevo levantamiento de datos para evaluar la percepción de los alumnos del sistema actual, y adquirir nociones que permitieran hacer un diseño centrado en el usuario de las nuevas funcionalidades. A partir de estos resultados se integraron los objetivos y preferencias de los alumnos de manera explícita en el sistema, y se agruparon en categorías funcionales y cualitativas las valoraciones de los alumnos.

   \par Luego se hizo un proceso de diseño de alto nivel a través del lenguaje de modelación \gls{i*}. Este se traspasó a un diseño basado en casos de uso a través de \acrshort{uml}. Finalmente, se produjo un diseño de funcionalidades de 3 partes que contempla: las preferencias y objetivos de los usuarios, las opciones en el sistema y, las funcionalidades específicas que permiten dar cumplimento a las dichas alternativas.

   \par Después, se procedió a un proceso de análisis y reestructuración del código actual, lo que permitió añadir nuevas funcionalidades, así como mejorar el sistema existente, asegurando su extensibilidad. Se solucionaron problemas de compatibilidad y \textit{bugs} en el código.

   \par Finalmente, se implementaron nuevas tecnologías como \textit{\gls{Celery}}, que permitieron la implementación de funcionalidades de suscripción personalizada. Permitiendo a cada alumno agregar recordatorios de los procesos habilitados.

   \par Este trabajo extiende las funcionalidades anteriores y reestructura el código existente, de manera de crear un sistema extensible y personalizable, favoreciendo la continuidad de este servicio. Logrando así los objetivos planificados.
   
\end{resumen}