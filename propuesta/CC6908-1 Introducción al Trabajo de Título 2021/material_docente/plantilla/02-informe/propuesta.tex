\documentclass[informe,guia]{upropuesta}

\title{El Título de mi Tema}

\author{Juan Peréz}
\guia{Juana Peréz, José Peréz}
\date{20XX}

\usepackage[T1]{fontenc}
\usepackage[utf8]{inputenc}

\usepackage[fixlanguage]{babelbib}
\usepackage{url}

\graphicspath{{imagenes/}}

\begin{document}

\maketitle

\begin{pauta}
Se debe quitar la guía antes de entregar el documento.

En la primera línea del \TeX: 

\begin{verbatim}
\documentclass[informe,guia]{upropuesta}
\end{verbatim}

Quitar la opción \texttt{guia}:

\begin{verbatim}
\documentclass[informe]{upropuesta}
\end{verbatim}

Además, hay que reemplazar los datos acá:

\begin{verbatim}
\author{Juan Peréz}
\guia{Juana Peréz, José Peréz}
\date{20XX}
\end{verbatim}

En comparación con la propuesta, se espera más detalles en todas las partes del documento (salvo en los objetivos), particularmente en la sección 2 (con el nuevo título ``\textbf{Estado del Arte}''), agregando además el plan de trabajo (algo que fue opcional en la propuesta), y el trabajo adelantado. 

Como una aproximación, se espera que el informe sea de 12--25 páginas.

[No hay que poner texto acá. Se puede empezar directamente con la introducción.]
\end{pauta}

\section{Introducción}\label{sec:intro}

\begin{pauta}
Dar una introducción al contexto del tema.

Explicar, en términos generales, el problema abordado.

Motivar la necesidad, la importancia y/o el valor, de tener una (mejor) solución.

[2--4 páginas]
\end{pauta}

\section{Estado del Arte}\label{sec:sa}

\begin{pauta}
Describe en más detalle la literatura, las técnicas, etc., relacionadas con el tema del trabajo.

[3--8 páginas con varias subsecciones]
\end{pauta}

Ejemplos de referencias:

\begin{itemize}
  \item Conferencia~\cite{CorlessJK97}
  \item Revista~\cite{NewmanT42}
  \item Tesis~\cite{Turing38}
\end{itemize}

% DBLP es una buena fuente de referencias en formata "BibTeX"
% en Ciencias de la Computación
% p.ej., https://dblp.org/pers/hd/t/Turing:Alan_M=

\section{Objetivos}\label{chap:obj}

\begin{pauta}
Describir las \textit{metas} del trabajo. Hay que contestar acá: ¿\textit{qué} quieres lograr? (La sección que sigue contestará la pregunta: ¿\textit{cómo} lo vas a lograr?)

Ejemplos de metas: lograr que X sea (más) eficiente, usable, seguro, completo, preciso, barato, informativo, posible por primera vez, etc.

Ejemplos de \textit{no} metas: implementar algo en Javascript, aplicar modelo Y sobre los datos, etc. (Estas cosas van en la descripción de la \textbf{Solución Propuesta}.)

Al final del trabajo, debería ser factible saber si se ha logrado los objetivos enumerados acá, o saber cuán bien se han logrado, o no. Por ejemplo, si la meta es tener algo eficiente en términos de tiempo, debería haber una forma de evaluar o estudiar los tiempos. Acá tendrás que definir la forma general en que se podrá evaluar el trabajo.

[No hay que poner texto acá. Se puede empezar directamente con el objetivo general.]
\end{pauta}

  \subsection*{Objetivo General}\label{sec:obj-g}
  
  \begin{pauta}
  Un \textit{resumen conciso} (no más de un párrafo) de la meta principal del trabajo, es decir, qué quieres lograr con el trabajo (o qué significa ``éxito'' en el contexto del trabajo).
  
  (``Titularse'' no es una repuesta válida. :))
  
  [1 párrafo]
  \end{pauta}

  \subsection*{Objetivos Específicos}\label{sec:obj-e}
  
  \begin{pauta}
  Una \textit{lista} de los hitos principales que se quieren lograr para (intentar) cumplir con el objetivo general. Divide el objetivo general en varios hitos que formarán las etapas del trabajo.
  
  Puede ser algo como: (1) obtener un conjunto de datos preciso, (2) diseñar un modelo de datos completo, (3) implementar un back-end que permita hacer tales consultas de forma eficiente, (4) diseñar e implementar una interfaz usable. Debería haber más detalles, pero no más de un párrafo por hito. 
  
  Los objetivos específicos deberían ``sumar'' al objetivo general.
  
  [Una lista 3--7 párrafos]
  \end{pauta}
  
  \begin{enumerate}
    \item ...
    \item ...
  \end{enumerate}
  
  \subsection*{Evaluación}\label{sec:eval}
  
  \begin{pauta}
  Describe cómo vas a poder evaluar el trabajo en términos de cuán bien  cumple con los objetivos planteados. Se pueden discutir los datos, las medidas, los usuarios, las técnicas, etc., utilizables para la evaluación.
  
  [1--2 párrafos]
  \end{pauta}

\section{Solución Propuesta}\label{sec:sol}

\begin{pauta}
Una descripción general de la solución propuesta: los datos, las técnicas, las tecnologías, las herramientas, los lenguajes, los marcos, etc., que se usarán para intentar lograr los objetivos planteados. Aquí hay que contestar la pregunta: ¿\textit{cómo} vas a lograr los objetivos planteados? Aquí, sí, está muy bien hablar de Javascript, CNNs, Numpy, Django, índices invertidos, árboles wavelet, privacidad diferencial, PageRank, Diffie--Hellman, triangulaciones de Delaunay, CUDA, Postgres, etc.

[2--4 páginas]
\end{pauta}


\section{Plan de Trabajo}\label{sec:pdt}

\begin{pauta}
Dar una lista de los pasos que se va a seguir para desarrollar la solución propuesta. La lista debería contemplar la evaluación del trabajo.

Para el informe final, se recomienda incluir un cronograma planificando el trabajo durante las 15 semanas del semestre del ``F'', haciendo referencia a los pasos listados en el plan de trabajo. Se puede extender el ejemplo dado, o usar otro software para generar el cronograma.

[1--2 páginas]
\end{pauta}

\begin{enumerate}
  \item ...
  \item ...
\end{enumerate}

\begin{ganttchart}{1}{15}
\gantttitle{Semana}{15} \\
\gantttitlelist{1,...,15}{1} \\
\ganttbar{Paso 1}{1}{3} \\
\ganttbar{Paso 2}{3}{7} \\
\ganttbar{...}{6}{8} \\
\ganttbar{Escribir la memoria}{4}{15}
\end{ganttchart}

\section{Trabajo Adelantado}\label{sec:tra}

\begin{pauta}
Describe el trabajo adelantado en la segunda parte del curso CC6908. \textit{El avance debería describir un trabajo práctico, que genere algún resultado preliminar pero novedoso, y que involucre aproximadamente 40 horas de trabajo. Es algo \textbf{adicional} a la profundización en las otras partes de la propuesta (como por ejemplo, en entender y describir el estado del arte).}

[2--5 páginas]
\end{pauta}

\bibliographystyle{babplain}
\bibliography{bibliografia}

\end{document}
